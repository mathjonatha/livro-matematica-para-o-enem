\chapter{Proporcionalidade}
Este tópico é fundamental, não apenas em Matemática, como na Matemática Financeira, na Física  na Química.Toda vez que trabalhamos com comparação de grandezas, algo comum na Física por exemplo, estamos utilizando os conceitos básicos da proporção.   

\section{Razão}
\begin{definition}
    Razão pode ser definida como o quociente (divisão ou fração) entre dois números reais, sendo o quociente do primeiro número com o segundo número. Desta forma, a razão entre os números $\mathrm{a}$ e $\mathrm{b}$ será o quociente:
\end{definition}

	
	\begin{center}
		$\frac{a}{b}$ ou $a:b$, para $(b \neq 0)$. 	Sendo lida da seguinte forma: 
		$\mathrm{a}$ está para $\mathrm{b}$
	\end{center}


	\begin{example}
	A razão $\frac{2}{3}$ é lida da seguinte forma: $2$ está para $3$.
	\end{example}

Ou seja, basicamente vamos ter numa razão a divisão entre um número real $\mathrm{a}$ e um número real $\mathrm{b}$ que resultará em um número real $\mathrm{c}$. Isto é:  

	\begin{center}
		$\frac{a}{b} = c$ 
	\end{center}
	
	\begin{example}
	   $\frac{3}{2} = 1,5$
	\end{example}

\section{Proporção}

\begin{definition}
    Proporção é definida como a igualdade entre duas ou mais razões. Isto é:
\end{definition}


	\begin{center}
	    $\frac{a}{b} = \frac{c}{d}$, para $(b$ e $d \neq 0)$
	\end{center}

Onde $\mathrm{a}$ e $\mathrm{d}$ são denominados extremos da proporção , $\mathrm{b}$ e $\mathrm{c}$ são os meios da proporção.

A leitura da proporção  $\frac{a}{b} = \frac{c}{d}$ é lida da seguinte forma: $\mathrm{a}$ está para $\mathrm{b}$, assim como $\mathrm{c}$ está para $\mathrm{d}$.

\begin{example}
    $\frac{3}{5}=\frac{6}{10}$, pois $\frac{3}{5}=0,6$ e $\frac{6}{10}=0,6$
\end{example}

\textbf{OBSERVAÇÃO:} Perceba, por exemplo, que se multiplicar por $2$ o primeiro lado da igualdade teremos que, necessariamente, multiplicar por $2$ o outro lado da igual, ou seja, estamos mantendo a proporcionalidade. Isto é:

\begin{example}
    $\frac{2 \cdot 3}{5}=\frac{2 \cdot 6}{10}$. Assim, manteremos a igualdade verdadeira. 
\end{example}

\subsection{Propriedades da Proporção}
    \begin{description}
    \item[1)] Em uma Proporção, o produto dos meios é igual ao produto dos extremos.
    \end{description}
 
 \begin{center}
    $\frac{a}{b} = \frac{c}{d} \Longleftrightarrow a\cdot d=b \cdot c$     
 \end{center}   

    \begin{description}
    \item[2)] Em uma Proporção, o produto dos meios é igual ao produto dos extremos. 
    \end{description}

\section{Escalas}

	\begin{itemize}
		\item Escala
		\subitem Definição de escala 
		\subitem Escala de ampliação
		\subitem Escala de redução
	\end{itemize}

\section{Porcentagem}
Toda a razão(termo explicado no capitulo anterior) que tem
como denominador o número $100$, dar-se o nome de razão centesimal. Por exemplo:
$$\frac{7}{100},~\frac{1}{100},~\frac{33}{100}$$

Veja outras formas de representação de razões centesimais:

$$\frac{7}{100}=0,07=7\% \textrm{ (se lê ``sete por cento")}$$
$$\frac{1}{100}=0,01=1\%\textrm{ (se lê ``um por cento")} $$
$$\frac{33}{100}=0,33=33\%\textrm{ (se lê ``trinta e três por cento")}$$

Essas expressões podem ser chamadas de taxas centesimais ou taxas percentuais. 

\begin{definition}
Porcentagem é o valor obtido ao aplicarmos uma taxa percentual a um determinado valor.

\end{definition}

\begin{example}
    João vendeu 60\% dos seus 100 cavalos. Quantos cavalos ele vendeu?
\end{example}

\begin{center}
    Para solucionar esse problema, devemos aplicar a taxa percentual (60\%) sobre o total de cavalos.
    $$60\%~de~100=\frac{60}{100}\cdot100=\frac{6000}{100}=60$$
    Portanto, João vendeu 60 cavalos dos 100 que ele tinha.
\end{center}

É possível encontra a porcentagem sendo utilizada em diversas situações como as de decréscimos, descontos, aumentos, acréscimos, diminuição, redução ou inflação podemos utilizar o \textbf{fator de multiplicação$(FM)$.}

\begin{definition}
O fator de multiplicação é diferente para aumento e desconto e a taxa percentual em ambos os casos sempre deverá ser um número decimal, ou seja, um número que possui vírgula. Veja as fórmulas referentes ao fator de multiplicação.

\end{definition}
\begin{center}
    Fator de multiplicação para aumento
    $$FM=1+taxa~de~aumento$$
    Fator de multiplicação para desconto
    $$FM=1-taxa~de~desconto$$
\end{center}
\begin{example}
    O salário-mínimo no ano de 2015 sofreu um aumento de 8,84\%. Sabendo que no ano de 2014 o salário-mínimo era de R\$ 724,00, qual será o valor do salário-mínimo para 2015?
\end{example}

\textbf{RESPOSTA:} Note que na situação esta ocorrendo um aumento, portanto usaremos a formula de taxa de aumento para resolver esse problema, veja:
$$FM=1+taxa~de~aumento$$
$$FM=1+8,84\%$$
$$FM=1+\frac{8,84}{100}$$
$$FM=1+0,0884$$
$$FM=1,0884$$

O valor em reais do salário-mínimo em 2015 será dado pelo produto do fator de multiplicação por R\$724,00.
$$ 724,00 \cdot 1,0884 = R\$ 788,00$$
Portanto, o valor do salario mínimo em 2015 será de $R\$788,00$

{\textbf{OBSERVAÇÃO:} Caso seja necessário utilizar a taxa de desconto, o procedimento é análogo ao feito acima, mudando apenas a formula de taxa de aumento para taxa de desconto.}
	



