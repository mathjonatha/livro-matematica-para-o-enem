\chapter{Sistemas de Unidades de Medidas}
(LEMBRAR DE COLOCAR AS IMAGENS)
\\O Sistema Métrico é um sistema internacional de medições que determina as unidades a serem utilizadas por cada uma das medidas, além de suas transformações. A unidade fundamental desse sistema é o metro ($m$), porém, cada medida tem sua unidade padrão, de acordo com o Sistema Internacional de Unidades. 

\section{Medidas de comprimento}
As medidas de comprimento são os mecanismos de medição mais utilizados. A unidade fundamental das medidas de comprimento é o metro ($m$). 
	\subsection{Múltiplos de metro}
		\begin{itemize}
			\item Quilômetro $(km) = 1000m = 10^3m$
			\item Hectômetro $(hm) = 100m = 10^2m$
			\item Decâmetro $(dam) = 10m$ 
			
		\end{itemize}

	\subsection{Submúltiplos de metro}
		\begin{itemize}
			\item Decimetro $(dm) = 0,1m = 10^{-1}m$
			\item centímetro $(cm) = 0,01m = 10^{-2}m$
			\item Milímetro $(mm) = 0,001m = 10^{-3}m$
		\end{itemize}
\section{Medidas de Superfície}
As medidas de superfície são as utilizadas na medição de
áreas. Sua unidade fundamental é o metro quadrado ($m^2$).
OBS: COLOCAR UM QUADRADO AQUI
	\subsection{Múltiplos de metro quadrado($m^2$)}
		\begin{itemize}
			\item Quilômetro quadrado $(km^2) = 1.000.000 m^2 = 10^6 m^2$
			\item Hectômetro quadrado $(hm^2) = 10.000 m^2 = 10^4 m^2$
			\item Decâmetro quadrado $(dam^2) = 100 m^2 = 10^2 m^2$
		\end{itemize}

	\subsection{Submúltiplos de metro quadrado}
		\begin{itemize}
			\item Decímetro quadrado $(dm^2) = 0,01m^2 = 10^{-2}m^2$
			\item centímetro quadrado $(cm^2) = 0,0001m^2 = 10^{-4}m^2$
			\item Milímetro quadrado $(mm^2)= 0,000001m^2 = 10^{-6}m^2$
		\end{itemize}
	
\section{Medidas Agrárias}
Dentre as medidas de superfícies, existem as medidas
agrárias, que são mais utilizadas para medir áreas rurais. Sua
unidade fundamental é o are ($a$).
	\begin{itemize}
		\item Centiare $(ca) = 1m^2=10^0m^2$
		\item Are $(a)= 100m^2 = 10^2m^2$
		\item Hectare $(ha) = 10000m^2 = 10^4m^2$
	\end{itemize}

\section{Medidas de Volume} 
As medidas de volume são utilizadas g
para medir o espaço ocupado por um
corpo tridimensional ou a sua
capacidade de armazenar alguma
substância. Em química, as medidas de
volume geralmente aparecem quando
medimos quantidades de líquidos. A
unidade métrica tradicional de volume
usada nesse caso é o litro (L). Em termos
do SI, um litro é definido exatamente
como 1 decímetro cúbico. 
	\subsection{Múltiplos do Metro Cubico:}
		\begin{itemize}
		    \item Quilômetro cúbico $(Km^3) = 1.000.000.000 m^3 = 10^9 m^3$
		    \item Hectômetro cúbico $(hm^3) = 1.000.000 m^3 = 10^6 m^3$
		    \item Decâmetro cúbico $(dam^3) = 1.000 m^3 = 10^3 m^3$
		\end{itemize}
	
	\subsection{Submúltiplos de metros cúbicos $(m^3)$}
		\begin{itemize}
		    \item Decímetro cúbico $(dm^3) = 0,001 m^3 = 10^{-3} m^3$
		    \item Centímetro cúbico $(cm^3) = 0,000001 m^3 = 10^{-6} m^3$
		    \item Milímetro cúbico $(mm^3) = 0,000000001 m^3 = 10^{-9} m^3$
		\end{itemize}
	
\section{Medidas de Capacidade}
As medidas de capacidade são utilizadas para representarem
as grandezas que definem o volume contido em um certo
reservatório. A mais utilizada é o litro ($l$).
	\\\textbf{Importante:}
	\begin{itemize}
	    \item $1l = 1dm^3$
	    \item $1 ml = 1 cm^3$
	    \item $10^3 l = 1 m^3$
	\end{itemize}

	\subsection{Múltiplos de litro:}
		\begin{itemize}
		    \item Quilolitro $(Kl) = 1000 l = 10^3 l$
		    \item Hectolitro $(hl) = 100 l = 10^3 l$
		    \item Decalitro $(dal) = 10 l = 10^1l$
		\end{itemize}
	
	\subsection{Submúltiplos de Litro:}
		\begin{itemize}
		    \item Decilitro $(dl) = 0,1 l = 10^{-1}l$
		    \item Centilitro $(cl) = 0,01 l = 10^{-2}l$
		    \item Mililitro $(ml) = 0,001 l = 10^{-3}l$
		\end{itemize}
	
\section{Medidas de Massa}
As medidas de massa são utilizadas para medir a quantidade
de massa em um corpo. No SI, a unidade fundamental de massa é
o quilograma ($kg$), embora o grama ($g$) seja a unidade mais
conveniente para a maioria das medidas de laboratório.
	\subsection{Múltiplos de Grama:}
		\begin{itemize}
		    \item Quilograma $(Kg) = 1000 g = 10^3 g$
		    \item Hectograma $(hg) = 100 g = 10^2 g$
		    \item Decagrama $(dag) = 10 g = 10^1 g$
		\end{itemize}
	
	\subsection{Submúltiplos de Grama:}
		\begin{itemize}
		    \item Decigrama $(dg) = 0,1 g = 10^{-1} g$
		    \item Centigrama $(cg) = 0,01 g = 10^{-2} g$
		    \item Miligrama $(mg) = 0,001 g = 10^{-3}$
		\end{itemize}
Existe ainda a unidade especial:
Tonelada $(t) = 1000 kg = 10^3 kg = 10^6 g$

\section{Exercícios}

Esta é um exemplo de exercício.
	\begin{exercise}
	Este é um bom lugar para fazer uma pergunta para testar o progresso da aprendizagem ou cimentar mais as idéias na mente dos alunos.
	\end{exercise}

