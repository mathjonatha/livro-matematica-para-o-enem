\chapter{Estatística}
\section{Introdução}
Neste capitulo estudaremos um assunto bastante abortado no enem, de início iremos entender como é um estudo estatístico e compreender suas etapas, logo após veremos o uso da matemática para organizar e extrair informações de diferentes formas acerca dos dados coletados. Antes veremos alguns conceitos básicos para melhor compreensão do tema, sendo o primeiro deles \textbf{medidas de posição} e \textbf{medidas de tendência central}.

\begin{align*}
(x+1)^3 &= (x+1)(x+1)(x+1) \\
&= (x+1)(x^2 + 2x + 1) \\
&= x^3 + 3x^2 + 3x + 1
\end{align*}

\subsection{Medidas de Posição}
\begin{definition}
	
\end{definition}
	\begin{itemize}
		\item Rol
		\item Classe
		\item Moda, Média e mediana 
	\end{itemize}