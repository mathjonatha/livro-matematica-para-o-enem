\chapter{Aritmética}
\textbf{Acho melhor falar o que é aritmética e depois de como o capitulo vai se desenvolver e talvez como surgiu os números}
Estudaremos nesta parte a base matemática que você necessita para da continuidade aos próximos tópicos. vamos iniciar te explicando conjuntos numéricos, logo após faremos uma breve revisão das operações básicas e a importância de compreender em qual conjunto estamos realizando as operações, vamos começar?
\section{Conjuntos dos Números Naturais}
O conjunto dos números naturais foi o primeiro que surgiu, representamos ele por $\mathbb{N}$ e assim como todos os outros ele é infinito, porém, possui um inicio. Veja a sua representação abaixo:

$$ \mathbb{N}= \{0,1,2,3,4,5,6,7,8,9,10,11, \ldots \}$$

Os detalhes acima que você precisa fixar são as chaves ($\{~~ \}$),que representa a ideia de conjunto, as reticências ($\ldots$), que passa a ideia de continuidade, nesse caso, que o conjunto continua infinitamente, por ultimo o zero ($0$) que demostra o inicio do conjunto.

Também podemos representar 
[FIGURA DO CONJUNTO DOS NÚMEROS NATURAIS]

\subsubsection{Subconjuntos dos Números Naturais}
Pense no subconjunto como sendo um grupo de alguns elementos pertencentes a um conjunto e que em geral esses elementos possuem alguma particularidade em comum ou excluem uma característica, vejamos alguns exemplos:

\begin{example}
	$$\mathbb{N}^{*}=\{1,2,3,4,5,\ldots,n,\ldots\}$$
	
	Por vezes, você pode encontrar esse mesmo subconjunto sendo representado por $\mathbb{N}-\{ 0 \}$, veja que é bem intuitivo ($conjunto~~dos~~naturais - 0$).
\end{example}


IMPORTANTE:
Conjunto vazio
O próprio conjunto dos naturais

\subsection{title}

\section{Operação com Números Naturais}
Agora que você já conhece o conjunto  e os subconjuntos dos números  naturais, veja como podemos operar com eles. Possivelmente você já conheça as operações básicas, porém vamos estudá-las somente dentro dos números naturais...afinal existem outros conjuntos numéricos que veremos futuramente...ops, spoiler.

\subsection{Adição}
A adição de dois números ou mais números naturais sempre resultará num número natural. Desta forma, o conjunto dos números naturais é fechado para a adição, ou seja, 
somando dois ou mais números naturais ainda estaremos dentro do conjunto dos números naturais.

\begin{example}
    $5+2=7$
\end{example}

Observe que os números 5 e 2 formam a parcela da adição, o símbolo $+$ indica a operação da adição e o 7 chamamos de soma. Portanto, adição é a operação onde obtemos a soma. 

\subsubsection{Propriedade Comutativa}
A ordem das parcelas não altera a soma, isto é:
\begin{center}
    $a+b=b+a=c$
\end{center}
\begin{example}
    $5+2=2+5=7$    
\end{example}


\subsubsection{Propriedade Associativa}
A soma de várias parcelas pode ser obtida reunindo-se duas a duas em qualquer ordem, isto é:
\begin{center}
    $a+b+c=a+(b+c)=(a+b)+c=d$
\end{center}

\begin{example}
    $3+7+1=3+(7+1)=(3+7)+1=11$  
\end{example}

\subsubsection{Elemento Neutro}
O número zero é o elemento neutro da adição, pois qualquer número adicionado com o zero resulta no próprio número, ou seja, não muda nada. Isto é:
\begin{center}
    $a+0=0+a=a$
\end{center}

\begin{example}
    $15+0=0+15=15$    
\end{example}

\subsection{Subtração}

Podemos dizer que a subtração é a operação inversa da adição (informação importante para o estudo da \textbf{álgebra ou equações}). Porém, o conjuntos dos números naturais não é fechado em relação a operação subtração, pois a subtração de dois números naturais nem sempre resulta em um número natural.

\begin{example}
    $5-2=3$
\end{example}

Observe que no exemplo acima o número 5 é o minuendo, o número 2 é o subtraendo, o símbolo $-$ indica a operação da subtração e o número 3 é a diferença. Portanto, subtração é a operação onde obtemos a diferença.

\subsubsection{A subtração não é comutativa}

\begin{center}
    $5-2=3$, porém $2-5=-3$, logo $5-2 \neq 2-5$  
\end{center}

\textbf{OBSERVAÇÃO:} Logo mais estudaremos o conjunto dos números inteiros. Este conjunto surgiu da necessidade de escrevermos os números negativos resultantes de subtrações como $2-5=-3$, o que não é possível utilizando somente os números naturais.

\subsection{Multiplicação}
O conjunto dos números naturais é fechado em relação a esta operação, pois a multiplicação de dois ou mais números naturais sempre resultará num número natural. Podemos imaginar inicialmente a multiplicação de números naturais como a operação associada a adição de parcelas iguais. 

\begin{example}
    $2 \cdot 14=28$
\end{example}

Observe que no exemplo acima o número $5$ é o minuendo, o número $2$ é o subtraendo, o símbolo $-$ indica a operação da subtração e o número $3$ é a diferença. Portanto, subtração é a operação onde obtemos a diferença.

\subsection{Divisão}


\subsection{Potenciação}


\subsection{Radiciação}

\subsection{}


\section{Conjunto dos Números Inteiros}
conjuntos numéricos
operações e 
operações em diferentes conjuntos

