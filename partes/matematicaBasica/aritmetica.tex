\chapter{Aritmética}
\section{Introdução}
Estudaremos nesta parte a base matemática que você necessita para da continuidade aos próximos tópicos, vamos iniciar te explicando conjuntos numéricos, logo após faremos uma breve revisão das operações básicas e a importância de compreender em qual conjunto estamos realizando as operações, vamos começar?
\subsection{Conjuntos dos Números Naturais}
O conjunto dos números naturais foi o primeiro que surgiu, representamos ele por $\mathbb{N}$ e assim como todos os outros ele é infinito, porém, possui um inicio. Veja a sua representação abaixo:

$$ \mathbb{N}= \{0,1,2,3,4,5,6,7,8,9,10,11, \ldots \}$$

Os detalhes acima que você precisa fixar são as chaves ($\{~~ \}$),que representa a ideia de conjunto, as reticências ($\ldots$), que passa a ideia de continuidade, nesse caso, que o conjunto continua infinitamente, por ultimo o zero ($0$) que demostra o inicio do conjunto.

Também podemos representar 
[FIGURA DO CONJUNTO DOS NÚMEROS NATURAIS]

\subsubsection{Subconjuntos dos Números Naturais}
Pense no subconjunto como sendo um grupo de alguns elementos pertencentes a um conjunto e que em geral esses elementos possuem algum particularidade em comum ou excluem uma característica, vejamos alguns exemplos:

\begin{example}
	$$\mathbb{N}^{*}=\{1,2,3,4,5,\ldots,n,\ldots\}$$
	
	Por vezes, você pode encontrar esse mesmo subconjunto sendo representado por $\mathbb{N}-\{ 0 \}$, veja que é bem intuitivo ($conjunto~~dos~~naturais - 0$).
\end{example}


IMPORTANTE:
Conjunto vazio
O próprio conjunto dos naturais

\subsection{title}


conjuntos númericos
operações e 
operações em diferentes conjuntos

