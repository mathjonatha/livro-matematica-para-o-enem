\chapter{Divisibilidade}
Sem duvidas, a divisão é a que os alunos sentem mais dificuldade das quatro operações básicas, mesmo ela esta tão presente em nossas vidas quantos as outras três. E de fato, sempre que você sai com os amigos para comer e divide a conta, esta usando divisão, ou quando prepara uma receita de bolinho para receber alguém especial e tem que dividir os ingredientes, também utiliza divisão e entre tantas outras coisas.

Em uma prova como o ENEM, saber dividir é primordial, pois a divisão pode se encaixar como pré-requisito de qualquer assunto, não só de matemática como também das disciplinas de ciências da natureza. Então se liga nesse capitulo que temos varias dicas legais para você.

\begin{definition}

 A divisibilidade, nada mais é do que a capacidade que a matéria tem de ser dividida em partes menores e iguais sem que sobre algo, ou matematicamente falando, se você pode dividir um número $\mathrm{P} $  por um número $\mathrm{Q}$, sem resto, dizemos que $\mathrm{P}$ é divisivel por $\mathrm{Q}$.
 \end{definition}
\section{Critérios de Divisibilidade}
Esses critérios vão te ajudar muito a melhorar o seu tempo de resolução de questões que envolvem divisão, com eles você poderá identificar se um numero é divisível por outro com mais facilidade.


Antes de iniciar, note que todo número é divisível por 1, veja: 

(FAZER IMAGEM DO ALGORITMO DE EUCLIDES PPROVANDO ISSO)
\subsection{Divisibilidade por $2$}
Para determinar se um dado número é divisível por $2$, precisamos apenas verificar se ele 
é par, isto é, todo número que termina em $0$, $2$, $4$, $6$ ou $8$ é divisível por $2$.


(TABELA DE DIVISIBILIDADE POR 2)