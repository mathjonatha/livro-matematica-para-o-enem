\chapter{Divisibilidade}
Sem duvidas, a divisão é a que os alunos sentem mais dificuldade das quatro operações básicas, mesmo ela esta tão presente em nossas vidas quantos as outras três. E de fato, sempre que você sai com os amigos para comer e divide a conta, esta usando divisão, ou quando prepara uma receita de bolinho para receber alguém especial e tem que dividir os ingredientes, também utiliza divisão e entre tantas outras coisas.

Em uma prova como o ENEM, saber dividir é primordial, pois a divisão pode se encaixar como pré-requisito de qualquer assunto, não só de matemática como também das disciplinas de ciências da natureza. Então se liga nesse capitulo que temos varias dicas legais para você.

\begin{definition}

 A divisibilidade, nada mais é do que a capacidade que a matéria tem de ser dividida em partes menores e iguais sem que sobre algo, ou matematicamente falando, se você pode dividir um número $\mathrm{P} $  por um número $\mathrm{Q}$, sem resto, dizemos que $\mathrm{P}$ é divisível por $\mathrm{Q}$.
 \end{definition}
\section{Critérios de Divisibilidade}
Esses critérios vão te ajudar muito a melhorar o seu tempo de resolução de questões que envolvem divisão, com eles você poderá identificar se um numero é divisível por outro com mais facilidade.


Antes de iniciar, note que todo número é divisível por 1, veja: 

(FAZER IMAGEM DO ALGORITMO DE EUCLIDES PROVANDO ISSO)
\subsection{Divisibilidade por $2$}
Para determinar se um dado número é divisível por $2$, precisamos apenas verificar se ele 
é par, isto é, todo número que termina em $0$, $2$, $4$, $6$ ou $8$ é divisível por $2$.


(TABELA DE DIVISIBILIDADE POR 2)

\subsection{Divisibilidade por $3$}
A regra de divisibilidade para o três, é um pouco mais complicada que a anterior, pois não teremos uma característica especifica no número para distinguir se é ou não divisível por $3$. Nesse caso, você precisa somar os dígitos do número de forma individual, por exemplo, a soma dos dígitos de $273$ é $2 + 7 + 3 = 12$ se $12$ for múltiplo de $3$, $273$ também é. Note que ao fazer isso, você terá um número menor que o anterior, logo fica mais fácil de saber se ele é múltiplo de $3$, caso continue com problemas, some os dígitos de novo e terá um numero menor ainda. 

(TABELA COM MÚLTIPLO DE 3)

.\subsection{Divisibilidade por $4$}
 Para saber se um número é divisível por $4$, você precisa ver se ele tem duas metades, ou seja, se ele pode ser dividido por $2$ duas vezes, por exemplo, o $48$ é divisível por $4$ pois é pá , logo o $2$ divide $48$ e tem $24$ como resultado, como $24$ é pá ainda pode ser dividido por $2$ o que satisfaz nosso critério.
 
Você pode ainda, usar outro critério que diz que se $abcd$ é divisível por $4$, ele terá os dois últimos dígitos sendo divisíveis por $4$, Ou seja, $cd$ é divisivel por $4$. 
(TABELA)
\subsection{Divisibilidade por $5$}
O critério do $5$ é bem simples, ele diz que, todo número que é divisível por $5$ tem o $5$ ou $0$ como ultimo digito, por exemplo $10$, $35$, $100$ e $2020$.
(TABELA)