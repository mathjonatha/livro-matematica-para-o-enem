\addcontentsline{toc}{chapter}{Habilidades e competências}
\chapter*{Habilidades e competências}
\section*{Competência de área 1}
    \subsection*{Construir significados para os números naturais, inteiros, racionais e reais}
    	\begin{description}
    		\item[H1 -]Reconhecer, no contexto social, diferentes significados e representações dos números e operações – naturais, inteiros,
    		racionais ou reais.
    		\item[H2 -]Identificar padrões numéricos ou princípios de contagem.
    		\item[H3 -]Resolver situação-problema envolvendo conhecimentos numéricos.
    		\item[H4 -]Avaliar a razoabilidade de um resultado numérico na construção de argumentos sobre afirmações quantitativas.
    		\item[H5 -]Avaliar propostas de intervenção na realidade utilizando conhecimentos numéricos.
    	\end{description}
	
\section*{Competência de área 2}
    \subsection*{Utilizar o conhecimento geométrico para realizar a leitura e a
    	representação da realidade e agir sobre ela.}
    	\begin{description}
    		\item [H6 -]Interpretar a localização e a movimentação de pessoas/objetos no espaço tridimensional e sua representação no espaço bidimensional.
    		\item [H7 -]Identificar características de figuras planas ou espaciais.
    		\item [H8 -]Resolver situação-problema que envolva conhecimentos geométricos de espaço e forma.
    		\item [H9 -]Utilizar conhecimentos geométricos de espaço e forma na seleção de argumentos propostos como solução de problemas do cotidiano.
    	\end{description}

\section*{Competência de área 3}
    \subsection*{Construir noções de grandezas e medidas para a compreensão da realidade
    	e a solução de problemas do cotidiano.}
    	\begin{description}
    		\item [H10 -]Identificar relações entre grandezas e unidades de medida.
    		\item [H11 -]Utilizar a noção de escalas na leitura de representação de situação do cotidiano.
    		\item [H12 -]Resolver situação-problema que envolva medidas de grandezas.
    		\item [H13 -]Avaliar o resultado de uma medição na construção de um argumento consistente.
    		\item [H14 -]Avaliar proposta de intervenção na realidade utilizando conhecimentos geométricos relacionados a grandezas e medidas.
    	\end{description}

\section*{Competência de área 4}
    \subsection*{Construir noções de variação de grandezas para a compreensão da realidade
    	e a solução de problemas do cotidiano.}
    	\begin{description}
    		\item [H15 -]Identificar a relação de dependência entrgrandezas.
    		\item [H16 -]Resolver situação-problema envolvendo a variação de grandezas, direta ou inversamente proporcionais.
    		\item [H17 -]Analisar informações envolvendo a variação de grandezas como recurso para a construção de argumentação.
    		\item [H18 -]Avaliar propostas de intervenção na realidade envolvendo variação de grandezas.
    	\end{description}

\section*{Competência de área 5}
    \subsection*{Construir noções de variação de grandezas para a compreensão da realidade
    	e a solução de problemas do cotidiano.}
    	\begin{description}
    		\item [H19 -]Identificar representações algébricas que expressem a relação entre grandezas.
    		\item [H20 -]Interpretar gráfico cartesiano que represente relações entre grandezas.
    		\item [H21 -]Resolver situação-problema cuja modelagem envolva conhecimentos algébricos.
    		\item [H22 -]Utilizar conhecimentos algébricos/geométricos como recurso para a construção de argumentação.
    		\item [H23 -]Avaliar propostas de intervenção na realidade utilizando conhecimentos algébricos.
    	\end{description}

\section*{Competência de área 6}
    \subsection*{Interpretar informações de natureza científica e social obtidas da leitura de gráficos e tabelas,
    	realizando previsão de tendência, extrapolação, interpolação e interpretação.}
    	\begin{description}
    		\item [H24 -]Utilizar informações expressas em gráficos ou tabelas para fazer inferências.
    		\item [H25 -]Resolver problema com dados apresentados em tabelas ou gráficos.
    		\item [H26 -]Analisar informações expressas em gráficos ou tabelas como recurso para a construção de argumentos.
    	\end{description}

\section*{Competência de área 7}
    \subsection*{Compreender o caráter aleatório e não determinístico dos fenômenos naturais e sociais e
    	utilizar instrumentos adequados para medidas, determinação de amostras e cálculos de probabilidade para
    	interpretar informações de variáveis apresentadas em uma distribuição estatística.}
    	\begin{description}
    		\item [H27 -]Calcular medidas de tendência central ou de dispersão de um conjunto de dados expressos em uma tabela de frequências de dados agrupados
    		(não em classes) ou em gráficos.
    		\item [H28 -]Resolver situação-problema que envolva conhecimentos de estatística e probabilidade.
    		\item [H29 -]Utilizar conhecimentos de estatística e probabilidade como recurso para a construção de argumentação.
    		\item [H30 -]Avaliar propostas de intervenção na realidade utilizando conhecimentos de estatística e probabilidade.
    	\end{description}

\vfill
\noindent Disponível em: \url{http://download.inep.gov.br/download/enem/matriz_referencia.pdf}. Acesso em: 06 de julho de 2020.